\chapter{Conclusion}\label{chap:4}
\begin{quote}
``And as the ends and ultimates of all things accord in some mean and measure with their inceptions and originals, that same multiplicit concordance which leads forth growth from birth accomplishing by a retrogressive metamorphosis that minishing and ablation towards the final which is agreeable unto nature so is it with our subsolar being.'' \\
--- James Joyce, \emph{Ulysses}
\end{quote}

This thesis presented two case studies.
In the first, we uncovered the nature of skimming through the lens of AZDWM inspectors.
We also discussed potential pitfalls for the fledgling empiricist.
The second case study was the design of a crowd-sourced skimmer detection application.
In this chapter, we addressed details of the Android API as well as system design.
Many of the features which made Bluetana practical in the field were not noted in the original paper.
The next two sections address the broader epistemological ground of this work.

\section{On Crowdsourced Application Design}

An individual observer cannot know everything: reality has a high degree of complexity.
Thus, collaboration, the implementation of crowd-sourced applications, is necessary for knowledge.
Crowd-sourcing can solve problems that need parallelism across space or time.
It is also a rich ground for that which is non-computational. 
These systems must cooperate with nontrivial-to-classify human input (chapter \ref{chap:2}).
This occurs through either the data they analyze or the system itself.
Thus, the design of a crowd sourced systems can lead to an understanding of human behavior.

\section{On Effective Data Analysis from Heterogeneous Sources}\label{sec:effective-data-analysis}

Information contained within this reality is non-homogeneous and avoids rigorous classification on account of the complexities generated via the combinatoric pairing of atomic deterministic systems that may be symbolically paralleled in logic. 
A large part of contemporary research is an attempt at a general solution to this ``problem''.
The closest we have come is the approximation of the human brain. 
However, approximation is not the only option.
It may be possible to build automated systems for doing the analysis presented in this paper.
These systems's implementation may also be a crucial epistemological step in humanity's development.
They would certainly save a lot of time.

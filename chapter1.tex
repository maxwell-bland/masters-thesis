\chapter{Introduction}\label{chap:1}
\begin{quote}
    ``In conclusion, \emph{Rhagoletis pomonella} may not be that unconventional after all.'' \\ 
---Jefferey Feder, \textit{The Apple Maggot Fly, Rhagoletis Pomonella}
\end{quote}
A skimming device typically consists of a microcontroller
attached to an EEPROM for recording credit card data exfiltrated from a Point-of-Sale
(POS) system~\cite{all_about_skimmers}. Prior work has developed mechanisms for the detection of external magnetic stripe 
based skimmers, which collect sensitive information by adding a second magnetic read head to the front
of the existing magnetic read head of a gas pump, ATM, or other payment terminal~\cite{scaife2018fear}. However, no work 
has been done to survey current law enforcement investigative mechanisms. Additionally, no work has been done 
to study the possibility of detecting internal skimmers, a variety installed within the casing of the POS system.
These skimmers exfitrate data through the use of a wireless module and collected by connecting into wires responsible
for the transmission of sensitive card information~\cite{gang_rigged_pumps_bluetooth}. This thesis provides both a survey of current skimmer 
investigative techniques and details the design of an application for the detection of internally installed
skimming devices. The resultant application, \emph{Bluetana}, was deployed across four states and more than thirty 
investigators and acts to ``crowdsource'' skimmer detection. So far, Bluetana has detected over
sixty skimming devices. Each skimmer makes around \$15,000 per day,\footnote{The typical card can be frauded 
for \$500, and the typical skimmer captures 30 cards per day.} indicating this research has prevented hundreds of 
thousands to millions of dollars in fraudulent charges \cite{bhaskar2019pay}.

    Chapter~\ref{chap:1} of this thesis addresses the issue of skimming and provides some background on current methodologies
for skimmer detection. Chapter~\ref{chap:2} focuses on a survey of current skimmer investigative techniques and provides
insight into the methods by which skimmers may be found. Chapter~\ref{chap:3} describes the implementation and design of 
an application for skimmer detection within the Android operating system. Chapter~\ref{chap:4} concludes and addresses the epistemological
space surrounding the collection of data from heterogenous sources and the development of crowdsourced applications.

\section{The Issue of Skimming}

While hacking a database might give adversaries access to password hashes eventually leading to a compromise of a bank account,
some opt for a physical route: skimming. The profitability of the installation of physical devices for fraud inside gas station 
sale terminals has driven a proliferation of the market, with 972 devices recovered from Florida and 148 in Arizona in 2018~\cite{bhaskar2019pay}.
Each of these devices can obtain around 30 unique cards, each the potential source of hundreds of dollars in fraud, per 
day~\cite{}. Despite the financial cost of these devices to card issuers and merchants, skimmers evade detection due both to
defenses added by criminals and the sparseness of the number of installations when compared to the total number of stations in 
a state. As we will see in Section~\ref{sec:eval-insp-beha}, reports from Arizona showed the percentage of stations in the state
hosting skimmers to be below \emph{X}\%. 

The typical consumer is also incapable of protecting themselves from internal skimmers without the leveraging of technology, as 
access to the internals of a pump is usually protected by a lock. It should be noted that this is not a problem for criminals 
since most Gilbarco Pump locks have a universal key\cite{}, and these pumps are installed in around 70\% of gas stations, a result
discovered via survey of 0.5\%  of the gas stations in the United States via Google Maps performed in the course of this research.

Manual inspection of pump internals without the assistance of technology is not effective
either, often taking around thirty minutes to complete with a success rate below 4\% \ref{sec:describing-inspections-skimmers}. 
Hence, there is an \emph{issue of skimming} which should be addressed, as it is funding a multi-million dollar economy on the backs of 
legitimate transactions, leading to market inefficiencies~\cite{guerra2003economics}. The main contribution of this thesis and related work is the 
demonstration of methods for internal skimmer detection based upon the tendency of criminals to opt for wireless exfiltration. 
Additionally, this thesis provides a qualitative comparison of data analysis methods in isolating skimmers.

At the current point in time, discussions with law enforcement have revealed that the majority of skimmers use Bluetooth modules. 
These discussions have also indicated that a ``mule", or hired hand, is then sent to connect and collect stolen card information 
from the device via Bluetooth.  This observation motivates the design and implementation of an application which uses features of 
Bluetooth for the detection of skimmers, hence the development of Bluetana. The efficacy of this application can serve as a 
starting point for continued research in the area of physical card fraud. 

\section{Methodologies of Detection}

Currently the primary method of skimmer detection is the via physical inspection of the gas pump itself. Inspectors will go
to gas stations either as a result of hinting on behalf of a credit company, law enforcement, or gas station owner and open
each pump to look for skimmers. This process is effective, although imperfect, as some skimmers are disguised or hidden within
the pump infrastructure itself, often covered in tape the same color as the ribbon wire they are connected to, such as in
Figure~\ref{fig:taped-skimmer}. In some cases during our testing of Bluetana, law enforcement were not able to find the skimmer
by physical means, although it was detected by the application, because the device had been hidden in a crevice of the machine.

\begin{figure}
  \centering
    \includegraphics[width=0.5\textwidth]{./pics/skimmerinpump.png}
  \caption{A skimmer taped the same color as the ribbon wire it is attached to.}
  \label{fig:taped-skimmer}
\end{figure}

There has been progress in the development of applications and tools for the detection of skimmers without extensive physical
inspection. In 2018, the Skimreaper tool was developed for detecting the external variety of skimmer via a detection of the secondary
read head that many external skimmers leverage \cite{scaife2018fear}. Additionally, some pump manufacturers are advertising alarm systems which
alert the owner and law enforcement when pump doors are opened \cite{flintloc}. Several applications for the detection of Bluetooth based
skimmers exist on the app market for Android currently, however, they do not leverage the extensive data analysis that Bluetana does,
and have been shown to be ineffective \cite{scaife2019rogue}. Skimmers using other wireless technologies, such as GSM, have also been found 
\cite{skimmer_GSM_enabled}, but discussions with law enforcement have indicated that these devices are much less common than Bluetooth technologies, and an extensive
listing of detection mechanisms for this species of devices is outside of the scope of this thesis.

\section{Contributions}

This thesis explores the motivation and development process of a Bluetooth-based skimmer detection mechanism. The primary contributions
of this work are a survey of current detection methodologies and a discussion of the implementation of the Bluetana application. It also
includes insights into the development of crowd-sourced applications and heterogeneous data analysis. This work has also had the practical
impact of discovering over sixty skimming devices and preventing hundreds of thousands to millions of dollars in fraud \cite{bhaskar2019pay}. The application
continues to see use by government inspectors across the United States.

    \subsection{Survey of Current Skimmer Detection Methodologies}

    Chapter~\ref{chap:2} contains a survey of hundreds of skimming inspection documents from the Arizona weights and measures department
    over the course of the 2016 through 2019 calendar years. It addresses the nature of the data and data acquisition, as well as internal
    classification methodologies for skimmer inspection documents. This heterogeneous data source required a relatively sophisticated
    technical pipeline to classify and understand, the details of which will be noted in Section~\ref{sec:tech-challenges}. An analysis
    of the data acquired will be noted in the sections following this, and will provide the motivation for the development of Bluetana.
    Discussion of the lessons learned during this process will be withheld for Section~\ref{sec:effective-data-analysis}.

    \subsection{Design and Implementation of Bluetooth Skimmer Detection}

    Chapter~\ref{chap:3} details the design of Bluetana as well as the systems which enable back-end data collection and analysis. It begins
    by describing the implementation of the application using the Android SDK, including the aspects of the Bluetooth protocol leveraged
    in the detection of skimmers. The chapter also describes the collection of data, features added to the application to make running
    the system more streamlined, and android application life-cycle management. This discussion will conclude with some of the open
    problems and drawbacks to using a crowdsourced android application in skimmer detection.

\section{Research Scope}

Skimming detection remains an open and important problem, and is the source of millions of dollars worth of fraud each year 
\cite{rippleshot2018fraud, bhaskar2019pay}. This work describes current methods of detection and proposes a novel space of exploration. 
The lessons contained herein can provide motivation for the next wave of research into the prevention and detection of these devices, ideally low cost, persistent wireless traffic monitors
which can be installed at a majority of gas stations world-wide. For now, Bluetooth-oriented skimmer detection remains a promising and effective
improvement over traditional physical inspection, and, along with other tools such as SkimReaper \cite{scaife2018fear}, can provide the first line of defense
for consumers, fuel station owners, and credit companies against the criminal element.
